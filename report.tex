\documentclass{article}

\usepackage[utf8]{inputenc}

\title{Rapport synthèse d'image}
\author{Alexandre Lewkowicz, Valentin Tolmer}

\begin{document}

\maketitle

\section{Introduction}

Notre projet consiste en un raytracer, basé sur la méthode de Monte-Carlo. Le
format d'entrée est du OBJ, donc la plupart des ressources de modèles sont
gérées.

La sortie est une image en PPM.

\section{Algorithme}

L'algorithme utilisé est celui de Monte-Carlo, avec 10000 rayons lancés pour
chaque point avec une distribution cosinus par rapport à la normale.

\section{Features}

\begin{itemize}
  \item Lumière ambiante, ponctuelle
  \item Lumière spéculaire
  \item Réflexion
  \item Interporlation des normales
  \item Réfraction
  \item Ombres portées
  \item Multi-threading
  \item Oct-tree pour accélérer le traitement.
  \item Parseur de fichier OBJ
  \item Anti-aliasing
  \item un début de gestion des textures et des normals map
\end{itemize}

De plus, nous avons un raytracer classique pour gérer les mirroirs.

\section{Critiques et améliorations}

Notre image souffre d'un effet "poivre et sel", inhérent à la méthode de
Monte-Carlo. Lancer beaucoup plus de rayons est une possibilité, mais elle est
très coûteuse (il faudrait augmenter de plusieurs ordres de magnitude le nombre
de rayons lancés par pixel).

Une méthode plus efficace pourrait être d'opérer
une passe de flou gaussien sur le résultat, ou de manière un peu similaire
d'effectuer du super-sampling. Un problème du flou gaussien serait que ça
rendrait les bords flous. Pour pallier à ce problème, un flou gaussien
sélectif, qui prend uniquement en compte les couleurs pas trop éloignées du
pixel concerné, serait utile. Idéalement, la distance serait mesurée en LAB car
l'effet est surtout dû à un problème d'intensité lumineuse.


On pourra également utiliser d'autre méthodes, comme la Bidirectional path
tracing, ou l'algorithme du Metropolis light transport.
Ces méthodes arrivent à calculer des chemins qui transporte plus d'information.
Par exemple, dans Metropolis, lorsque un chemin est valide, il est réutilisé.


\end{document}
